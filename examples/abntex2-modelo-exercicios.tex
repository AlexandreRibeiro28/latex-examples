%% abntex2-modelo-exercicios.tex
%% Copyright 2013 A. A. Lucas
%
% This work may be distributed and/or modified under the
% conditions of the LaTeX Project Public License, either version 1.3
% of this license or (at your option) any later version.
% The latest version of this license is in
%   http://www.latex-project.org/lppl.txt
% and version 1.3 or later is part of all distributions of LaTeX
% version 2005/12/01 or later.
%
% This work has the LPPL maintenance status `maintained'.
% 
% The Current Maintainer of this work is A. A. Lucas.
%
% This work consists of the files abntex2-modelo-exercicios.tex

% ------------------------------------------------------------------------
% ------------------------------------------------------------------------
% Modelo de Lista de Exercícios
% * Utiliza a classe abntex2 para formatação.
% ------------------------------------------------------------------------
% ------------------------------------------------------------------------
\documentclass[12pt,a4paper,oneside]{abntex2}

% PACOTES

% Pacotes fundamentais 
\usepackage{cmap}	% Mapear caracteres especiais no PDF
\usepackage{lmodern}	% Usa a fonte Latin Modern			
\usepackage[T1]{fontenc}	% Selecao de codigos de fonte.
\usepackage[utf8]{inputenc}	% Codificacao do documento (conversão automática dos acentos)
\usepackage{lastpage}	% Usado pela Ficha catalográfica
\usepackage{indentfirst}	% Indenta o primeiro parágrafo de cada seção.
\usepackage{color}	% Controle das cores
\usepackage{graphicx}	% Inclusão de gráficos
\usepackage{amsmath}	% Pacote para equações multi linhas

% Pacotes adicionais
\usepackage{lipsum}	% para geração de dummy text

% Comandos adicionais
\providecommand{\imprimirdisciplina}{}
\newcommand{\disciplina}[1]{\renewcommand{\imprimirdisciplina}{#1}}

% Informações do aluno e documento
\titulo{Lista de Exercícios}
\autor{Nome do aluno}
\disciplina{Nome da disciplina}
\data{2013}
\orientador{Nome do professor}

% Configurações de aparência do PDF final
\definecolor{blue}{RGB}{41,5,195} % alterando o aspecto da cor azul

% informações do PDF
\makeatletter
\hypersetup{
     	%pagebackref=true,
		pdftitle={\@title}, 
		pdfauthor={\@author},
    	pdfsubject={\@title},
	    pdfcreator={LaTeX with abnTeX2},
		pdfkeywords={abnt}{latex}{abntex}{abntex2}{lista de exercícios}, 
		colorlinks=true,       		% false: boxed links; true: colored links
    	linkcolor=blue,          	% color of internal links
    	citecolor=blue,        		% color of links to bibliography
    	filecolor=magenta,      		% color of file links
		urlcolor=blue,
		bookmarksdepth=4
}
\makeatother

\setlength{\parindent}{0cm} % Remove indentação dos parágrafos
\setlength{\parskip}{0.5cm} % Espaçamento entre parágrafos

% Novo estilo de cabeçalho com a identificação do aluno
\makepagestyle{cabidaluno}
	% Cabeçalhos
	\makeoddhead{cabidaluno}
		{\imprimirautor}{\imprimirtitulo}{\thepage}
	\makeevenhead{cabidaluno}
		{\imprimirautor}{\imprimirtitulo}{\thepage}
	\makeheadrule{cabidaluno}{\textwidth}{\normalrulethickness} % linha separadora
	% Rodapé
	\makeoddfoot{cabidaluno}
		{\imprimirorientador\ - \imprimirdisciplina, \imprimirdata}{}{}
	\makeevenfoot{cabidaluno}
		{\imprimirorientador\ - \imprimirdisciplina, \imprimirdata}{}{}


% INICIO DO DOCUMENTO

\begin{document}
	
	% Elementos textuais
	\textual
	\pagestyle{cabidaluno} % Inclusão de cabeçalhos e rodapés personalizados
	
1) Dessa maneira, responda os exercícios da lista sem preocupações com formatação ou outras distrações que tomam seu precioso tempo.

2) Você pode inserir tabelas caso seja necessário, como a tabela \ref{tab:exemplo}.

\begin{table}[htb]
	\center
	\begin{tabular}{|l|c|c|}
		\hline
		\textbf{Marca} & \textbf{Satisfeitos} & \textbf{Insatisfeitos} \\ \hline
		A & $74,71\%$ & $12,36\%$ \\ \hline
		B & $79,72\%$ & $11,75\%$ \\ \hline
		C & $23,75\%$ & $61,88\%$ \\ \hline
		D & $12,16\%$ & $75,68\%$ \\ \hline
	\end{tabular}
	\caption{Tabela exemplo.}
	\label{tab:exemplo}
\end{table}

3) Insira equações simples:
\[
	P(E|Erro) = \frac{P(E \cap Erro)}{P(Erro)}
\]

E até outras mais complexas com alinhamento em colunas e quebras de linha:
% Equação com várias linhas
\begin{align*}
	\begin{split}
		P(Erro) &= P[(A \cap Erro) \cup (B \cap Erro) \cup (C \cap Erro) \cup (D \cap Erro) \cup (E \cap Erro)] \\
		& = P(A \cap Erro) + P(B \cap Erro) + P(C \cap Erro) + P(D \cap Erro) + P(E \cap Erro) \\
		& = P(A) \cdot P(Erro|A) + P(B) \cdot P(Erro|B) + P(C) \cdot P(Erro|C) + \\
		& P(D) \cdot P(Erro|D) + P(E) \cdot P(Erro|E)		
	\end{split}
\end{align*}

4) Aproveite as funcionalidades do \LaTeX, organize suas ideias em listas.
\begin{itemize}
	\item Ideia 1
	\item Ideia 2
\end{itemize}

5) Faça como quiser, o \LaTeX formata para você.

\end{document}